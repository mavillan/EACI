\documentclass[spanish, fleqn]{article}
\usepackage[spanish]{babel}
\usepackage[utf8]{inputenc}
\usepackage{amsmath}
\usepackage{amsfonts,txfonts}
\usepackage{mathrsfs}
\usepackage[colorlinks, urlcolor=blue]{hyperref}
\usepackage{fourier}
\usepackage[top = 2.5cm, bottom = 2cm, left = 2cm, right = 2cm]{geometry}
\usepackage{graphicx}




\title{\textsc{Elementos de Análisis para Computación e Informática} \\
	\textsc{Examen Final}}

\author{Martín Villanueva}
\date{27 de Mayo 2016}

\begin{document}
\maketitle


\section*{\textsc{Tareas}}


\begin{description}

	\item[\textsc{Tarea 1.}] Sean $(X , \sigma)$ y $(X, \sigma')$ espacios métricos y \textit{quasimétricos} respectivamente. Denotemos a la topología generada/inducida por $\sigma$ por:
	\begin{align*}
		\mathcal{F} = \{ A \subset X: \ A \ \text{es un abierto, i.e:} \ \ \forall x \in A, \ \exists \delta>0: \ B_{\sigma}(x,\delta) \subset A \}
	\end{align*}
	donde adicionalmente sabemos que una base para esta topología es:
	\begin{align*}
		B = \{ B_{\sigma}(x, \sigma): \ x \in X, \ \delta \ \in \mathbb{R}_{0}^{+}  \}
	\end{align*}
	y también consideremos la siguiente \textit{posible} base para dicha topología:
	\begin{align*}
		B' = \{ B_{\sigma'}(x,\delta): \ x \in X, \ \delta \ \in  \mathbb{R}_{0}^{+} \cup \{ \infty \} \}
	\end{align*}
	se puede ver claramente que $B \subset B'$, pues $B'$ contiene los mismos miembros que $B$, más las bolas de radio infinito (\textit{que abarcan todo el espacio!}). Además como $B$ es una base para $\mathcal{F}$, cualquier abierto $\in \mathcal{F}$ puede escribirse como una reunión de miembros de $B$.

	Sea $\mathcal{A}$ un conjunto indexador (no necesariamente contable, ni finito), entonces cualquier reunión de bolas \newline $ \bigcup_{\alpha \in \mathcal{A}} B_{\alpha}  \ (B_{\alpha} \in B)$, puede formarse también con miembros de $B'$, pues $B \subset B'$. 
	Consideremos ahora reuniones sobre $B'$. Si entre los miembros no hay bolas de radio infinito, el resultado es análogo 
	al anterior. Consideremos entonces el caso de una reunión con al menos una bola de radio infinito $B_{\sigma'}(x_0, \infty)$:
	\begin{align*}
		\bigcup_{\alpha \in \mathcal{A}} B_{\alpha} \ (B_{\alpha} \in B') = B_{\sigma'}(x_0, \infty)
	\end{align*}
	tal resultado puede obtenerse por reuniones de miembros en $B$ como sigue:
	\begin{align*}
		\bigcup_{\delta > 0} B_{\sigma}(x_0, \delta)
	\end{align*}

	Luego, hemos probado que cada reunión de miembros en una base, puede formarse con los miembros de la otra (y viceversa). Por lo tanto ambas son bases de $\mathcal{F}$, y entonces $\sigma$ y $\sigma'$ generan/inducen la misma topología.




	\item[\textsc{Tarea 2.}] Se definen $(S,\sigma)$ un espacio métrico, el conjunto $E \neq \emptyset$, el conjunto de funciones $\mathcal{F}(E,S) = \left\{f: E \rightarrow S \right\}$, y la aplicación:
	\begin{align*}
		\Sigma(f,g) = \sup_{t \in E}\ \sigma \left(f(t),g(t)\right), \ \ \ \ \text{con} \ \ f,g \in \mathcal{F}(E,S),
	\end{align*}
	donde se quiere probar que $(\mathcal{F}(E,S),\Sigma)$ es un espacio \textit{quasimétrico}. Ya que $\sigma \left(f(t),g(t)\right)$ $(\forall t \in E)$ puede no estar acotado superiormente, entonces $\Sigma(f,g)$ puede ser infinita. Se verifica que $\Sigma$ cumple los axiomas usuales para una métrica, para $f,g,h \in \mathcal{F}(E,S$:
	\begin{enumerate}
	 	\item $\left(\Sigma(f,g) \geq 0 \right)$. Se sigue directamente desde el codominio de $\sigma : S \times S \rightarrow \mathbb{R}_0^+$.
	 	\item $\left(\Sigma(f,g) = 0 \leftrightarrow f=g \right)$. Como $\Sigma(f,g) = 0 \leftrightarrow \displaystyle \sup_{t \in E} \  \sigma\left(f(t), g(t) \right) = 0$, por la definición de supremo esto es equivalente a $\sigma\left( f(t), g(t) \right) \leq 0$. Ya que $\sigma$ es un métrica de $S$, la única posibilidad es que $\sigma\left(f(t),g(t)\right)=0 \leftrightarrow f(t) = g(t)$.
	 	\item $\left(\Sigma(f,g) = \Sigma(g,f) \right)$. Dada la simetría de la métrica $\sigma$, entonces:
	 	\begin{align*}
	 		\Sigma(f,g) = \sup_{t \in E} \ \sigma\left(f(t), g(t) \right) = \sup_{t \in E} \ \sigma\left(g(t), f(t) \right) = \Sigma(g,f).
	 	\end{align*}
	 	\item $\left(\Sigma(f,g) \leq \Sigma(f,h) + \Sigma(h,g) \right)$. Si se fija un $t_0 \in E$, luego para $f(t_0), g(t_0), h(t_0) \in S$ se cumple:
	 	\begin{align}
	 		\sigma\left(f(t_0), g(t_0) \right) \leq \sigma\left(f(t_0), h(t_0) \right) + \sigma\left(h(t_0), g(t_0) \right),
	 	\label{eq:triang}
	 	\end{align}
	 	pues $\sigma$ es métrica. Como (\ref{eq:triang}) es válida $\forall t_0 \in E$, entonces se verifica que:
	 	\begin{align*}
	 		\sup_{t\in E} \ \sigma\left(f(t), g(t) \right) \leq \sup_{t\in E} \ \sigma\left(f(t), h(t) \right) + \sup_{t\in E} \ \sigma\left(h(t), g(t) \right).
	 	\end{align*}
	\end{enumerate}
	Dado que $\Sigma: \mathcal{F}(E,S) \times \mathcal{F}(E,S) \rightarrow [0,\infty]$ y verifica los axiomas de métrica, entonces $\left(\mathcal{F}(E,S), \Sigma \right)$ es un espacio \textit{quasimétrico}.



	\item[\textsc{Tarea 3.}]




	\item[\textsc{Tarea 4.}] Veamos cada punto separadamente:
	\begin{enumerate}
		\item $\displaystyle \int_{\mathbb{R}} |\chi_{[-1,1]}(t)|\ dt = \int_{-1}^{1} |1| dt = 2 < \infty$, por lo tanto $\chi_{[-1,1]} \in L^1$.
		\item $\displaystyle \widehat{\chi_{[-1,1]}(t)} = \int_{\mathbb{R}} \chi_{[-1,1]}(t) e^{-2 \pi i \omega t} dt = \int_{-1}^{1} e^{-2 \pi i \omega t} dt$. Haciendo uso de la identidad de Euler:
		\begin{align*}
			\int_{-1}^{1} e^{i (-2 \pi \omega t)} dt = \int_{-1}^{1} \cos(-2 \pi \omega t) + i \sin(-2 \pi \omega t) dt
		\end{align*}
		tomando en cuenta la paridad de las funciones (\textit{y que la integral de una función impar en un intervalo simétrico es nula}):
		\begin{align*}
			\ldots = \int_{-1}^{1} \cos(2 \pi \omega t) dt - i \underbrace{\int_{-1}^{1} \sin(2 \pi \omega t) dt]}_{= 0 \ \text{por imparidad}} = \frac{\sin(2 \pi \omega t)}{2 \pi \omega} \bigg|_{-1}^{1} = \frac{\sin(2 \pi \omega)}{\pi \omega}
		\end{align*}
		\item

	\end{enumerate}




	\item[\textsc{Tarea 5.}] Veamos por partes:
	\begin{enumerate}
		\item Denotemos como $\displaystyle f(t)=e^{-t^2}$ y $\displaystyle g(t)=(1+t^2)^{-1}$. En primer lugar, es fácil notar que las derivadas de $f(t)$ son la misma función multiplicadas por un polinomio (\textit{regla de la cadena}). Cada nueva derivada aumenta en un grado el grado del polinomio respectivo, pues la derivada del exponente es $-2t$.
		Luego se puede escribir $f^{(n)}(t)=f(t)P_n(t)$, con $P_n$ el polinomio de grado $n$ respectivo. Entonces:
		\begin{align*}
			\lim_{|t|\rightarrow \infty} t^p D^q f(t) = \lim_{|t|\rightarrow \infty} t^p P_q(t) f(t) = \lim_{|t|\rightarrow \infty} \widehat{P}_{p+q}(t) f(t) = \lim_{|t|\rightarrow \infty} \frac{\widehat{P}_{p+q}(t)}{e^{t^2}} = 0 \ \ \ \forall \ p,q \in \mathbb{N}_0,
		\end{align*}
		lo último debido a que la exponencial crece más rápido que cualquier polinomio en $t \rightarrow \infty$. En segundo lugar, para probar que $g \notin \mathcal{C}^{\infty}(\mathbb{R},\mathbb{C})$, basta mostrar que existe alguna combinación de $p,q \in \mathbb{N}_0$ tales que $t^p D^q g(t) \rightarrow 0$ no se satisface. Veamos para $p=3$ y $q=1$:
		\begin{align*}
			\lim_{|t|\rightarrow \infty} \frac{t^3}{1+t^2} \rightarrow \infty,
		\end{align*}
		lo que completa la demostración.

		\item Utilizando la notación de multi-índices para $\alpha,\beta \in \mathbb{N}_0^n$, se quiere probar que los siguientes dos conjuntos:
		\begin{align}
			\mathcal{C}_{\downarrow}^{\infty}(\mathbb{R}^n,\mathbb{C}) &= \left\{ f \in \mathcal{C}^{\infty}(\mathbb{R}^n,\mathbb{C}): x^{\alpha} D^{\beta} f(x) \rightarrow 0, \ \ \text{cuando} \ \ ||x||_2 \rightarrow 0, \ \ \ \forall \alpha,\beta \in \mathbb{N}_0^n \right\} \label{eq:def1} \\
			\mathcal{S}(\mathbb{R}^n,\mathbb{C}) &= \left\{ f \in \mathcal{C}^{\infty}(\mathbb{R}^n,\mathbb{C}): ||f ||_{\alpha, \beta} = \sup_{x \in \mathbb{R}^n} \left|x^{\alpha} D^{\beta} f(x)\right| < \infty, \ \ \ \forall \alpha,\beta \in \mathbb{N}_0^n \right\} \label{eq:def2}
		\end{align}
		corresponden a definiciones equivalentes del mismo conjunto. Partamos de $\mathcal{C}_{\downarrow}^{\infty}(\mathbb{R}^n,\mathbb{C})$. Desde la definición del límite, notar que $\forall \epsilon$ existe un $M(\epsilon)$ tales que:
		\begin{align*}
			| x^{\alpha} D^{\beta} f(x)| < \epsilon, \ \ \text{cuando} \ \ ||x||_2 < M(\epsilon).
		\end{align*}
		Adicionalmente puesto que  $x^{\alpha} D^{\beta} f$ es continua en $[-M,M]$, por el Teorema de Weierstrass \cite{Weier}, esta debe alcanzar su máximo en tal intervalo. Entonces:
		\begin{align*}
			|x^{\alpha} D^{\beta} f(x)| \leq \max \left\{ \epsilon, \max_{x\in [-M(\epsilon),M(\epsilon)]} x^{\alpha} D^{\beta} f(x) \right\} < \infty, \ \ \ \forall x \in \mathbb{R}^n,
		\end{align*}
		que es equivalente a (\ref{eq:def2}). Partamos ahora de $\mathcal{S}(\mathbb{R}^n,\mathbb{C})$ (el cual existe y es finito). En primer lugar, denotemos a $\displaystyle \sup_{x \in \mathbb{R}^n} \left|x^{\alpha} D^{\beta} f(x)\right| = C_{\alpha,\beta}$. Si se define $\alpha+1 =(\alpha_1+1, \alpha_2+2, \cdots, \alpha_n+1)$, entonces:
		\begin{align*}
			{Si} \ \ \  \sup_{x \in \mathbb{R}^n} \left|x^{\alpha+1} D^{\beta} f(x)\right| < \infty
			\Rightarrow \left|x^{\alpha+1} D^{\beta} f(x)\right| \leq C_{\alpha+1,\beta} < \infty
			\Rightarrow  \left|x^{\alpha} D^{\beta} f(x)\right| \leq \frac{C_{\alpha+1,\beta}}{|x_1 \cdots x_n|},
		\end{align*}
		para todo $x \in \mathbb{R^n}$. Y luego (usando el Teorema de Acotamiento \cite{Squeeze}):
		\begin{align*}
			\lim_{||x||_2 \rightarrow 0} \frac{C_{\alpha+1,\beta}}{|x_1 \cdots x_n|} = 0 \Longrightarrow x^{\alpha} D^{\beta} f(x)\rightarrow 0 \ \ \ \text{cuando} \ \ \ ||x||_2\rightarrow 0,  \ \ \ \forall \alpha,\beta \in \mathbb{N}_0^n,
		\end{align*}
		que es la la definición en (\ref{eq:def1}).

		\item

		\item

		\item
	\end{enumerate}




	\item[\textsc{Tarea 6.}] Sea $f \in \mathcal{C}_{\downarrow}^\infty(\mathbb{R},\mathbb{C})$, se quiere demostrar que:
	\begin{align}
		(2 \pi i \gamma)^p D^q \widehat{f}(\gamma) = \left[D^p (-2 \pi i x)^q f(x)\right]\widehat{} \ \ \ \ \ \forall p,q \in \mathbb{N}_0.
	\label{eq:teo}
	\end{align}
	Para ello se definen las siguientes dos proposiciones:
	\begin{align*}
		\mathbf{P}_1(p) &: \ \ (2 \pi i \gamma)^p \widehat{f}(\gamma) = \left[D^p f(x) \right]\widehat{} \ \ \ \ \ \ \ \forall p \in \mathbb{N}_0 \\
		\mathbf{P}_2(q) &: \ \ D^q \widehat{f}(\gamma) = [(-2 \pi i x)^q f(x)]\widehat{} \ \ \ \ \ \forall q \in \mathbb{N}_0,
	\end{align*}
	cuyos casos bases (derivados en \cite{DymMcKean}) son respectivamente:
	\begin{align}
		(p=1) &: \ \ (2 \pi i \gamma) \widehat{f}(\gamma) = \left[ D f(x) \right]\widehat{} \label{eq:base1} \\
		(q=1) &: \ \ D \widehat{f}(\gamma) =  \left[ (-2 \pi i x) f(x) \right]\widehat{} \label{eq:base2}.
	\end{align}
	Probemos cada una por inducción. Para $\mathbf{P}_1$ se tiene:
	\begin{align*}
		(2 \pi i \gamma)^{p+1} \widehat{f}(\gamma) &= (2 \pi i \gamma)(2 \pi i \gamma)^{p} \widehat{f}(\gamma) \\
		&= (2 \pi i \gamma) \left[ D^p f(x) \right]\widehat{} \\
		&\stackrel{\text{(\#)}}{=} \ \ \left[ D(D^p f(x)) \right]\widehat{} \\
		& = \left[ D^{p+1} f(x) \right]\widehat{},
	\end{align*}
	donde en (\#) se uso al hipótesis inductiva (\ref{eq:base1}). Se procede de igual modo para $\mathbf{P}_2$:
	\begin{align*}
		D^{q+1} \widehat{f}(\gamma) &= D\left[D^{q} \widehat{f}(\gamma)\right] \\
		&= D_{\gamma} \left[(-2 \pi i x)^q f(x) \right]\widehat{} \\
		&\stackrel{\text{(\#)}}{=} \ \ \left[(-2 \pi i x)\ (-2 \pi i x)^q f(x) \right]\widehat{} \\
		&= \left[(-2 \pi i x)^{q+1} f(x) \right]\widehat{},
	\end{align*}
	donde en (\#) se uso la hipótesis inductiva (\ref{eq:base2}). Como $\mathbf{P}_1(p)\Rightarrow \mathbf{P}_1(p+1)$,
	y $\mathbf{P}_2(q) \Rightarrow \mathbf{P}_2(q+1)$, se han demostrado entonces ambas proposiciones. Luego, se puede proceder como sigue:
	\begin{align*}
		(2 \pi i \gamma)^p D^q \widehat{f}(\gamma) \ \ \stackrel{\mathbf{P}_2}{=} \ \ (2 \pi i \gamma)^p \left[ (-2 \pi i x)^q f(x) \right]\widehat{} \ \ \stackrel{\mathbf{P}_1}{=} \ \ \left[ D^p (-2 \pi i x)^q f(x) \right]\widehat{}\ \ ,
	\end{align*}
	lo que completa la prueba de (\ref{eq:teo}).

	Para la segunda parte, notar que aplicando valor absoluto a (\ref{eq:teo}) se obtiene:
	\begin{align*}
		\left|\gamma^p D^q \widehat{f}(\gamma)\right| &= \frac{1}{(2 \pi)^p} \left| \int_{\mathbb{R}} (-2 \pi i)^q D_{x}^{p} x^q e^{-2 \pi i \gamma x} dx \right| \\
		&\leq (2 \pi)^{q-p} \int_{\mathbb{R}} \left| D_{x}^{p} x^q f(x) \right| dx \\
		&= (2 \pi)^{q-p} \ \ ||D_{x}^p x^q f(x)||_{L^1} < \infty,
	\end{align*}
	puesto que $f \in \mathcal{C}_{\downarrow}^\infty(\mathbb{R},\mathbb{C}) \Rightarrow  D^p x^q f \in \mathcal{C}_{\downarrow}^\infty(\mathbb{R},\mathbb{C})$ y que $\mathcal{C}_{\downarrow}^\infty(\mathbb{R},\mathbb{C}) \subset L^1(\mathbb{R},\mathbb{C})$. Entonces como $\widehat{f} \in \mathcal{C}^\infty(\mathbb{R},\mathbb{C})$ y satisface $\left|\gamma^p D^q \widehat{f}(\gamma)\right| < \infty$, se concluye que $\widehat{f} \in \mathcal{C}_{\downarrow}^\infty(\mathbb{R},\mathbb{C})$, i.e, la transformada de Fourier mapea $\mathcal{C}_{\downarrow}^\infty(\mathbb{R},\mathbb{C})$ en sí mismo.




 
	\item[\textsc{Tarea 7.}] Se listan a continuación algunos ejemplos de funciones $f \in \mathcal{C}_{C}^{\infty}(\mathbb{R},\mathbb{C})$ (la extensión a varias variables se muestra a continuación):
	\begin{enumerate}
		\item Un ejemplo clásico es la función \textit{bump}, definida como:
		\begin{align*}
		\psi(x) =
		\begin{cases}
		e^{\frac{1}{x^2-1}} & |x|<1 \\
		0 & |x|>= 1,
		\end{cases}	
		\end{align*}
		cuya grafica se ve en Figura \ref{fig:cbump}. Esta puede ser fácilmente extendida a otros invervalos de compacidad:
		\begin{align*}
		\widehat{\psi}(x) =
		\begin{cases}
		e^{\frac{1}{(x-a)^2-1}} & |x|<r \\
		0 & |x|>= r,
		\end{cases}	
		\end{align*}
		que tiene soporte compacto en $[a-r, a+r]$ con $a,r \in \mathbb{R}$. 

	\begin{figure}[htpb!]
	\centering
	\includegraphics[width=10cm]{bump}
	\caption{Gráfico de función \textit{bump} clásica.}
	\label{fig:cbump}
	\end{figure}

	\item Otra construcción interesante considera a la función $\displaystyle f(x) = e^{\frac{-1}{x}}$, cuya ``gracia'' es que está en $\mathcal{C}^{\infty}(\mathbb{R})$ y $f(0)=0$. Con esta es posible definir:
	\begin{align}
		\phi(x) = \frac{f(x)}{f(x)-f(1-x)}
	\label{eq:trick}
	\end{align}
	la cual tambien pertenece a $\mathcal{C}^{\infty}(\mathbb{R})$, pero ademas $\phi(0)=0$ y $\phi(1)=1$. Entonces una función de soporte compacto se puede construir como sigue:
	\begin{align*}
		\psi(x) =
		\begin{cases}
		0 & x \leq 0 \\
		\phi(x) & 0 < x <1 \\
		1 & 1 \leq x \leq a \\
		\phi(-(x-[a+1])) & a < x < a+1 \\
		0 & x \geq a+1
		\end{cases}	
	\end{align*}
	donde $a \in \mathbb{R}\ (> 1)$. Que básicamente consiste en poner la ``copia'' invertida de $\phi$ (respecto al eje $y$) y trasladada a la derecha en $a+1$, y definiendo el valor de la función en el intervalo intermedio $[1,a]$ como $1$ para asegurar la continuidad. En la Figura \ref{fig:const} se muestra esta contrucción con $a=2$.
	\begin{figure}[htpb!]
	\centering
	\includegraphics[width=10cm]{construction}
	\caption{Construcción de función de soporte compacto.}
	\label{fig:const}
	\end{figure}
	
	Notar que la misma construcción puede realizarse siguiendo el mismo procedimiento y usando (\ref{eq:trick}), pero con
	alguna otra función $f$ que cumpla $f \in \mathcal{C}^{\infty}(\mathbb{R})$ y $f(0)=0$.

	\end{enumerate}

	Por último, dada una función $\psi \in \mathcal{C}_{C}^{\infty}(\mathbb{R},\mathbb{C})$ con soporte $[a,b] \ \text{y} \ a,b \in \mathbb{R}$, esta puede ser fácilmente extendida a una función $\in \mathcal{C}_{C}^{\infty}(\mathbb{R}^n,\mathbb{C})$, bajo la siguiente construcción:
	\begin{align}
	\Phi(x_1,x_2,\ldots,x_n) = \psi(x_1)\ \psi(x_2)\ \cdots \ \psi(x_n),
	\label{eq:ndbump}
	\end{align}
	cuyo soporte compacto es $[a,b]\times \stackrel{n-2 \text{ veces}}{\ldots} \times [a,b]=[a,b]^n$. En la Figura \ref{fig:3dbump} se muestra tal construcción para la función \textit{bump} clásica en dos variables.
	
	\begin{figure}[htpb!]
	\centering
	\includegraphics[width=10cm]{3dbump}
	\caption{Gráfico de función \textit{bump} clásica en dos variables.}
	\label{fig:3dbump}
	\end{figure}



	\item[\textsc{Tarea 8.}] Sea $f \in \mathcal{C}_{\downarrow}^{\infty}(\mathbb{R},\mathbb{C})$ una función con soporte compacto $\text{supp}(f) \subset [a,b] \subset [0,T]$ ($0<a<b<T$), con la respectiva prolongación periódica (de periodo $T$)
	\begin{align}
	 	g(x) = \sum_{n \in \mathbb{Z}} f(x - n T), \ \ \ \ x \in \mathbb{R}.
	 	\label{eq:prolongacion}
	\end{align} Considerando la norma uniforme $||f||_{\infty} := \sup_{x \in \mathbb{R}} |f(x)|$. Como ya probamos anteriormente (Weierstrass):
	\begin{align*}
		f \in \mathcal{C}_{\downarrow}^{\infty}(\mathbb{R},\mathbb{C}) \Rightarrow 0 \leq ||f||_{\infty} < \infty.
	\end{align*}
	Debido a la construcción de la prologación periódica (\ref{eq:prolongacion}), se sabe que no existen traslapes entre
	las \textit{copias} de $f$ en los distintos periodos. Entonces:
	\begin{align*}
		||g||_{\infty} = \sup_{x \in \mathbb{R}} |g(x)| = \sup_{x \in \mathbb{R}} \left|\sum_{n \in \mathbb{Z}} f(x - n T)\right| \stackrel{(\#)}{=} \sup_{x \in \mathbb{R}} |f(x-nT)|\ \ (\forall n \in \mathbb{Z})= \sup_{x \in \mathbb{R}} |f(x)| = ||f||_{\infty} < \infty 
	\end{align*}
	donde $(\#)$ es posible, pues como se hizo notar, todas las copias de $f$ son iguales y sin traslapes. Puesto que $||g||_{\infty} < \infty$, se comprueba entonces la convergencie de la serie.




	\item[\textsc{Tarea 9.}] 

\end{description}


\bibliographystyle{plain}
\bibliography{referencias}


\end{document}
